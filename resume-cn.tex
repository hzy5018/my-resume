% !TEX TS-program = xelatex
% !TEX encoding = UTF-8 Unicode
% !Mode:: "TeX:UTF-8"

\documentclass{resume}
\usepackage{zh_CN-Adobefonts_external} % Simplified Chinese Support using external fonts (./fonts/zh_CN-Adobe/)
% \usepackage{NotoSansSC_external}
% \usepackage{NotoSerifCJKsc_external}
% \usepackage{zh_CN-Adobefonts_internal} % Simplified Chinese Support using system fonts
\usepackage{linespacing_fix} % disable extra space before next section
\usepackage{cite}


% \setlist[itemize,1]{label=$\times$}
% \setlist[itemize,2]{label=$\checkmark$}
% \setlist[itemize,3]{label=$\diamond$}
% \setlist[itemize,4]{label=$\bullet$}


\begin{document}
\pagenumbering{gobble} % suppress displaying page number

\name{胡志源}

\basicInfo{
  \email{huzhy5018@gmail.com} \textperiodcentered\ 
  \phone{(+86) 182-218-61528} \textperiodcentered\ 
  \linkedin[Hu Zhiyuan]{https://www.linkedin.com/in/hu-zhiyuan-1ba50798/}}
 
\section{\faGraduationCap\  教育背景}
\datedsubsection{\textbf{上海复旦大学}, 上海}{2011.09 -- 2014.07}
\textit{理学硕士}\ 基础数学(GPA 3.6/4.0)\\
\textit{主修课程}:
现代常微分方程,物理学与偏微分方程,应用偏微分方程,控制理论基础
\datedsubsection{\textbf{兰州大学}, 兰州, 甘肃}{2007.08 -- 2011.07}
\textit{学士}\ 数学与应用数学(GPA 4.01/4.5)\\
\textit{主修课程}:
数学模型,数值分析,运筹学,图论,金融工程

\section{\faUsers\ 工作经验}
\datedsubsection{\textbf{口碑-商业智能中心}}{2019.05至今}
\role{资深数据分析师}{}
\begin{onehalfspacing}
  \begin{itemize}
    \item \textbf{专题分析}
    \begin{itemize}
      \item[*] 广告专题分析:(1)口碑\&支付宝、阿里妈妈广告指标体系搭建;(2)广告投放效果分析;(3)广告人群画像分析,优化广告投放人群;
      \item[*] 搜索推荐优化:(1)搜索推荐业务指标体系搭建;
      \item[*] 内容优化分析:(1)短视频、日记、直播等内容相关的指标体系搭建;(2)短视频、日记、直播等内容的引导转化效率分析;
      \item[*] 旺铺售卖模型:(1)旺铺售卖过程指标体系建立;(2)基于随机森林搭建旺铺售卖模型,给出售卖策略建议以及挑选关键旺铺因子;(3)旺铺的销售效率提升25\%; 
      \item[*] 人效管理分析: (1) 负责构建人效指标体系;(2)负责搭建人效专题分析看板;
    \end{itemize}
    \item \textbf{数据产品:} (1) 负责搭建生活服务数字化体系;(2)负责构建生活服务事业部数据门户;(3)负责建设生活服务事业部中高频、商业化、消费医疗模块数据看板系统;(4)双11、双12实时数据看板产品;
    \item \textbf{数据治理:} (1) 统一和管理生活服务行业口径; (2) 设计和建设生活服务垂直域数据(dws层或adm层);
  \end{itemize}
\end{onehalfspacing}


\datedsubsection{\textbf{数灏智能科技有限公司}}{2018.01 -- 2019.03}
\role{算法工程师}{}
\begin{onehalfspacing}
\begin{itemize}
  \item 债券违约模型:(1)从0到1构建债券违约特征工程;(2)用tensorflow实现债券违约统计模型;(3)以随机森林作为基线模型,构造债券违约机器学习模型;(4)模型准确率提高到80\%, KS 值提高到45;
  \item 多因子模型:(1)构建资产的影响因子数据库;(2)用Lasso、随机森林,筛选对于资产比较重要的因子;	 
  \item 财务造假模型:(1)构建财务指标体系以及指标库;(2)异常分析, 用isolate forest算法检测财务造假数据点;(3)以模型调用API的方式,为券商提供数据服务;
  % \item 构建财务模型,包含估值、情绪等模型;并API服务化,对外提供数据模型服务
  % \item 构建债券、宏观、大宗商品、行业、个股因子筛选模型(Lasso、ElasticNet、RandomForest);并且建立相应的web api,对外提供数据模型服务.	  
\end{itemize}
\end{onehalfspacing}

\datedsubsection{\textbf{上海行邑信息科技有限公司}}{2016.03 -- 2018.01}
\role{数据挖掘工程师}{}
\begin{onehalfspacing}
\begin{itemize}
  \item \textbf{某国企反欺诈数据挖掘项目}:该公司反欺诈服务可行性分析,\textbf{并且该项目通过客户公司验收}。上海某国企征信公司(甲方)计划开发平台征信数据,为平台内p2p、现金贷、小额贷款公司提供反欺诈服务。由我司进行反欺诈数据挖掘服务的可行性研究,
  论证该平台数据能够用于反欺诈,能够开发出相应的反欺诈服务产品。论证内容主要有:1)数据质量是否能够支撑数据反欺诈服务;2)评分卡模型可行性;
  3)图挖掘模型可行性,论证是否能够选取合适字段,构造网络. 主要子项目如下:
  \begin{itemize}
     \item[*] ETL与特征工程
     \item[*] 图挖掘或网络挖掘模型: (1)构建异构网络、同构网络;(2)在同构网络上,应用PageRank算法;(3)构建网络的一度、二度模型
     \item[*] 社团挖掘:基于社团挖掘算法Louvain算法,构造P2P金融反欺诈团伙欺诈模式挖掘模型    
     \item[*] 评分卡模型:(1)以Logistic Regression为基础,建立评分卡模型;(2)模型AUC = 0.8, K-S值为45\%
  \end{itemize}	
  \item {\textbf{关联图谱系统项目}}:关联图谱基于图数据库建立关系网络图,是一种可视化的智能分析产品。通过数据抽取和转换,图计算引擎对数据进行查询和分析,
实现秒级数据运算和数据可视化,并以图谱的形式展示给用户的图形分析工具。用户可以基于已建好的图谱进行查询、分析和探索。
从功能上,分为图分析和图计算。图分析主要基于图数据库(janusgraph),提供在线实时分析;图计算利用分布式计算框架(Spark GraphX),计算全图
的PageRank和社团,同时计算节点的一度特征、二度特征计算。主要子模块如下:
\begin{itemize}
  \item[*] 数据集成:开发图生成协议,适配用户数据.
  \item[*] 在线分析模块:基于图数据库,进行在线实时分析.
  \item[*] 离线分析模块:用Spark GraphX 图计算:主要包括社团发现(Louvain算法),PageRank;用hive进行预计算,统计图特征.    
  \item[*] 系统管理模块:用户管理、权限管理等.
\end{itemize}
  \item DFP(Device FingerPrint: 设备指纹) 算法: (1)重构设备指纹系统指标体系;(2)特征工程优化与重造;(3)优化以及调整模型参数;(4)调整模型结构;(5)算法工程实现;(6)设备指纹算法系统在测试环境(POC)中,准确率达100\%。
  \item 游戏防代充模型:(1)数据统计分析,确定代充欺诈判断规则,为数据打标;(2)特征工程;(3)以Logistics Regression算法为基础,构建防代充模型;(3)模型的F2-score 95\%, 同时为该公司减少5\%坏账率。
\end{itemize}
\end{onehalfspacing}

\datedsubsection{\textbf{携程旅行网}}{2014.10 -- 2016.03}
\role{数据挖掘工程师}{}
\begin{onehalfspacing}
\begin{itemize}
  \item 用户行为模型:(1)分析用户价格尾数(0,4,5,9)的敏感性;(2)A/B test实验:将部分房型价格尾号调整为0,4,5,9;(3)实验结果是实验组比对照组提高了1\%。
  \item 品牌溢价模型:(1)构造马甲房型,并且保证自营与马甲之间的差价,实现收益与产量最大化,通过做A/B test,得到不同房型的最优价差;在最优价差下,酒店收益提高了8\%左右的收益。
  \item 自动调价模型:(1)分析酒店产品的售卖进度和库存之间的关系,动态调整酒店产品价格;(2)在定价策略执行阶段,酒店收益每月能提高了6\%左右。
  \item 产量预测模型:(1)用Xgboost,构建产量预测模型,并筛选出有用的因子;(2)产量预测结果,直接指导业务部门将如何与酒店谈合作,降低公司从酒店方面拿客房的成本
\end{itemize}
\end{onehalfspacing}

% Reference Test
%\datedsubsection{\textbf{Paper Title\cite{zaharia2012resilient}}}{May. 2015}
%An xxx optimized for xxx\cite{verma2015large}
%\begin{itemize}
%  \item main contribution
%\end{itemize}

% \section{\faUsers\ 项目经验}
% \datedsubsection{\textbf{反欺诈咨询项目}}{上海行邑科技有限公司(2017.10--2018.01)}
% \role{项目角色:负责人}{}
% 项目介绍:上海某国企征信公司(甲方)计划开发平台征信数据,为平台内p2p、现金贷、小额贷款公司提供反欺诈服务。由我司进行反欺诈数据挖掘服务的可行性研究,
% 论证该平台数据能够用于反欺诈,能够开发出相应的反欺诈服务产品。论证内容主要有:1)数据质量是否能够支撑数据反欺诈服务;2)评分卡模型可行性;
% 3)图挖掘模型可行性,论证是否能够选取合适字段,构造网络
% \\[2pt]
% \begin{onehalfspacing}
% 主要子项目如下:
%   \begin{itemize}
%      \item 数据清洗与融合分析
%      \item 数据质量评估	     
%      \item 标注样本数据
%      \item 特征工程
%      \item 图挖掘或网络挖掘: 构建异构网络、同构网络;在同构网络上,应用PageRank算法;构建网络的一度、二度特征
%      \item 应用社团挖掘算法Louvain算法,挖掘社团	     
%      \item 构建基于Logistic Regression的评分卡模型
%   \end{itemize}
% \end{onehalfspacing}

% 项目通过甲方企业的中期验收以及最终项目验收.
% \datedsubsection{\textbf{关联图谱系统项目}}{自研项目(2018.10--现在)}
% \role{项目角色:负责人}{}
% 项目介绍:
% 关联图谱基于图数据库建立关系网络图,是一种可视化的智能分析产品。通过数据抽取和转换,图计算引擎对数据进行查询和分析,
% 实现秒级数据运算和数据可视化,并以图谱的形式展示给用户的图形分析工具。用户可以基于已建好的图谱进行查询、分析和探索。
% 从功能上,分为图分析和图计算。图分析主要基于图数据库(janusgraph),提供在线实时分析;图计算利用分布式计算框架(Spark GraphX),计算全图
% 的PageRank和社团,同时计算节点的一度特征、二度特征计算。
% \\[2pt]
% \begin{onehalfspacing}
% 主要子模块如下:
% \begin{itemize}
%   \item 数据集成:开发图生成协议,适配用户数据.
%   \item 在线分析模块:基于图数据库,进行在线实时分析.
%   \item 离线分析模块:用Spark GraphX 图计算:主要包括社团发现(Louvain算法),PageRank;用hive进行预计算,统计图特征.    
%   \item 系统管理模块:用户管理、权限管理等.
% \end{itemize}
% \end{onehalfspacing}

\section{\faCogs\ IT 技能}
% increase linespacing [parsep=0.5ex]
\begin{itemize}[parsep=0.5ex]
   \item 编程语言(按熟悉程度从高到低排序)
     \begin{itemize}
      \item[*] 大数据分析:HQL(Hive SQL)/SQL
      \item[*] 数据分析与模型分析:Python(Numpy,Scipy,Matplotlib,Seaborn, Pandas), R
      \item[*] 大数据算法实现:Scala/Java(Spark应用程序)
      \item[*] Web服务:Go
      \item[*] 其他:C/C++      
     \end{itemize}		     
  \item 熟悉机器学习算法
  \begin{itemize}
  \item[*] 监督学习:LASSO,LAR,Logistic Regression,朴素贝叶斯,决策树,随机森林,KNN,SVM,Xgboost(lightGBM), 神经网络;
  \item[*] 无监督学习:K-Means, PCA,GMM(Gaussian Mixture Model),HMM(Hidden Markov Model), 层次聚类,异常监测;时间序列分析等
  \end{itemize}
  \item 熟悉关系型数据库: SQL Server 2008, Mysql, No-sql: MongoDB, 图数据库: Neo4j, janusgraph, hugegraph, 内存数据库: Redis	  
  \item 熟悉分布式计算,熟练分析平台Hadoop,Spark
  \item 熟悉神经网络计算平台,如tensorflow, mxnet	  
  \item 平台: Linux, Docker
  \item 开发: Spark, Web Service
\end{itemize}

\section{\faHeartO\ 荣誉与获奖}
\datedline{\textit{一等奖}, 复旦大学奖学金}{2013.09}
\datedline{\textit{一等奖}, "高教杯"全国大学生数学建模大赛}{2010.12}
\datedline{其他奖项, 国家励志奖学金}{2010.09}
\datedline{其他奖项, 在SCI(Journal of Mathematics and Physics)上发表论文: Pullback attractors for a nonautonomous nonclassical diffusion equations with time delay}{2011.06}
\section{\faInfo\ 其他}
% increase linespacing [parsep=0.5ex]
\begin{itemize}[parsep=0.5ex]
  %\item 技术博客: http://blog.yours.me
  %\item GitHub: https://github.com/username
  %\item 语言: 英语 - 熟练(TOEFL xxx)
  \item 语言: 英语 - 一般, 普通话 - 母语
\end{itemize}

%% Reference
%\newpage
%\bibliographystyle{IEEETran}
%\bibliography{mycite}
\end{document}
