% !TEX TS-program = xelatex
% !TEX encoding = UTF-8 Unicode
% !Mode:: "TeX:UTF-8"

\documentclass{resume}
\usepackage{zh_CN-Adobefonts_external} % Simplified Chinese Support using external fonts (./fonts/zh_CN-Adobe/)
% \usepackage{NotoSansSC_external}
% \usepackage{NotoSerifCJKsc_external}
% \usepackage{zh_CN-Adobefonts_internal} % Simplified Chinese Support using system fonts
\usepackage{linespacing_fix} % disable extra space before next section
\usepackage{cite}

\begin{document}
\pagenumbering{gobble} % suppress displaying page number

\name{胡志源}

\basicInfo{
  \email{huzhy5018@gmail.com} \textperiodcentered\ 
  \phone{(+86) 182-218-61528} \textperiodcentered\ 
  \linkedin[Hu Zhiyuan]{https://www.linkedin.com/in/hu-zhiyuan-1ba50798/}}
 
\section{\faGraduationCap\  教育背景}
\datedsubsection{\textbf{上海复旦大学}, 上海}{2011.09 -- 2014.07}
\textit{理学硕士}\ 基础数学(GPA 3.6/4.0)\\
\textit{主修课程}:
现代常微分方程,物理学与偏微分方程,应用偏微分方程,控制理论基础
\datedsubsection{\textbf{兰州大学}, 兰州, 甘肃}{2007.08 -- 2011.07}
\textit{学士}\ 数学与应用数学(GPA 4.01/4.5)\\
\textit{主修课程}:
数学模型,数值分析,运筹学,图论,金融工程

\section{\faUsers\ 工作经验}
\datedsubsection{\textbf{锋涛资产管理有限公司} 上海}{2018.01 -- 现在}
\role{数据科学家}{}
\begin{itemize}
  \item 构建、评估、部署债券违约模型,并将模型准确率提高到80\%, KS 值提高到45.
  \item 研究量化金融基本面机器学习模型,并且构建、评估、部署债券、宏观、大宗商品、行业、个股机器学习模型.
  \item 设计Web产品API,线上产品系统
  \item 设计、构建实时流处理模块,使其能进行数据清洗,转化,进一步计算特征并且进行模型预测.	  
\end{itemize}

\datedsubsection{\textbf{上海行邑信息科技有限公司}}{2016.03 -- 2018.01}
\role{数据科学家}{}
\begin{onehalfspacing}
\begin{itemize}
  \item 作为团队负责人,负责某国企反欺诈数据挖掘项目,项目目的主要是验证该公司数据能够用于机器学习,并且该项目通过客户公司验收. 主要子项目如下:
  \begin{itemize}
     \item 数据清洗与融合分析
     \item 特征工程
     \item 图挖掘或网络挖掘: 构建异构网络、同构网络;在同构网络上,应用PageRank算法;构建网络的一度、二度特征
     \item 构建基于Logistic Regression的评分卡模型
  \end{itemize}	
  \item 重新为POC(Proof of Concept)设计DFP(Device FingerPrint: 设备指纹) 算法,使其满足POC要求: 准确率达到100\%.
  \item 改进生产系统设备指纹算法, 调整DFP(Device FingerPrint: 设备指纹) 算法参数,并且开发DFP(Device FingerPrint: 设备指纹)模型算法部分.
  \item 设计某游戏公司防代充模型,并且达到95\% 准确率, 为该公司减少5\%坏账率.
  \item 设计ETL(Etract, Transform, Load) 系统,并且实现数据可视化功能,以提高售前、销售部门的效率.	  
\end{itemize}
\end{onehalfspacing}

\datedsubsection{\textbf{携程旅行网}}{2014.10 -- 2016.03}
\role{数据挖掘工程师}{}
\begin{onehalfspacing}
\begin{itemize}
  \item 用户行为分析,对用户进行差异化定价
  \item 品牌溢价分析,最大限度榨取品牌价值	  
  \item 根据酒店产品的售卖进度和库存之间的关系,动态调整酒店产品价格;在定价策略执行阶段,酒店收益每月能提高了8\%
  \item 产量预测,用Xgboost,构建产量预测模型,并筛选出有用的因子;产量预测结果,直接指导业务部门将如何与酒店谈合作,降低公司从酒店方面拿客房的成本
\end{itemize}
\end{onehalfspacing}

% Reference Test
%\datedsubsection{\textbf{Paper Title\cite{zaharia2012resilient}}}{May. 2015}
%An xxx optimized for xxx\cite{verma2015large}
%\begin{itemize}
%  \item main contribution
%\end{itemize}

\section{\faCogs\ IT 技能}
% increase linespacing [parsep=0.5ex]
\begin{itemize}[parsep=0.5ex]
  \item 编程语言: HQL(Hive SQL)/SQL, Python(Numpy,Scipy,Matplotlib,Seaborn, Pandas), R, Scala/Java, Go, C++/C.
  \item 熟悉机器学习算法,监督学习:LASSO,LAR,Logistic Regression,朴素贝叶斯,决策树,随机森林,KNN,SVM,Xgboost, 神经网络;无监督学习:K-Means, PCA,GMM(Gaussian Mixture Model),HMM(Hidden Markov Model), 层次聚类,异常监测;时间序列分析等
  \item 熟悉关系型数据库: SQL Server 2008, Mysql, No-sql: MongoDB, 图数据库: Neo4j, 内存数据库: Redis	  
  \item 熟悉分布式计算,熟练分析平台Hadoop,Spark
  \item 熟悉神经网络计算平台,如tensorflow, mxnet	  
  \item 平台: Linux
  \item 开发: Spark, Web Service
\end{itemize}

\section{\faHeartO\ 荣誉与获奖}
\datedline{\textit{一等奖}, 复旦大学奖学金}{2013.09}
\datedline{\textit{一等奖}, "高教杯"全国大学生数学建模大赛}{2010.12}
\datedline{其他奖项, 国家励志奖学金}{2010.09}
\datedline{其他奖项, 在SCI(Journal of Mathematics and Physics)上发表论文: Pullback attractors for a nonautonomous nonclassical diffusion equations with time delay}{2011.06}
\section{\faInfo\ 其他}
% increase linespacing [parsep=0.5ex]
\begin{itemize}[parsep=0.5ex]
  %\item 技术博客: http://blog.yours.me
  %\item GitHub: https://github.com/username
  %\item 语言: 英语 - 熟练(TOEFL xxx)
  \item 语言: 英语 - 一般, 普通话 - 母语
\end{itemize}

%% Reference
%\newpage
%\bibliographystyle{IEEETran}
%\bibliography{mycite}
\end{document}
